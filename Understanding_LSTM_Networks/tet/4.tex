The key to LSTMs is the cell state, the horizontal line running through the top of the diagram.

The cell state is kind of like a conveyor belt. It runs straight down the entire chain, with only some minor linear interactions. It’s very easy for information to just flow along it unchanged.

\begin{figure}[h]
	\centering
	\includegraphics[width=0.75\textwidth]{fig/7.png}
\end{figure}

The LSTM does have the ability to remove or add information to the cell state, carefully regulated by structures called gates.

Gates are a way to optionally let information through. They are composed out of a sigmoid neural net layer and a pointwise multiplication operation.

\begin{figure}[h]
	\centering
	\includegraphics[width=0.15\textwidth]{fig/8.png}
\end{figure}

The sigmoid layer outputs numbers between zero and one, describing how much of each component should be let through. A value of zero means “let nothing through,” while a value of one means “let everything through!”

An LSTM has three of these gates, to protect and control the cell state.
